\documentclass{bioinfo}
\copyrightyear{2020} \pubyear{2020}

\access{Advance Access Publication Date: Day Month Year}
\appnotes{Manuscript Category}

\begin{document}
\firstpage{1}

\subtitle{Structural Bioinformatics}

\title[MolSysMT]{MolSysMT: A computational laboratory bench to work with biomolecular systems in Python.}
\author[D. Prada-Gracia \& L.M. Moreno-Vargas]{Diego Prada-Gracia$^{\text{\sfb 1,}*}$ and Liliana M. Moreno-Vargas$^{\text{\sfb 1,}*}$}
\address{$^{\text{\sf 1}}$Computational Biology and Drug Design Research Unit, Mexico City
Children's Hospital Federico Gómez, Mexico City, 06720, Mexico.}

\corresp{$^\ast$To whom correspondence should be addressed.}

\history{Received on XXXXX; revised on XXXXX; accepted on XXXXX}

\editor{Associate Editor: XXXXXXX}

\abstract{\textbf{Summary:} Bla, bla, bla.\\
\textbf{Availability and implementation:} The open source code is available at the
\href{https://github.com/uibcdf/MolSysMT}{uibcdf/MolSysMT GitHub repository} to be used, modified
and shared according to the MIT License. A conda package with the latest stable version is distributed through the
\href{https://anaconda.org/uibcdf/molsysmt}{"uibcdf" conda channel}. And installation
instructions, user's documentation and jupyter notebooks with illustrative examples are available
at \href{http://www.uibcdf.org/MolSysMT}{http://www.uibcdf.org/MolSysMT}.\\
\textbf{Contact:} \href{prada.gracia@gmail.com}{prada.gracia@gmail.com},
\href{lm.moreno.vargas@gmail.com}{lm.moreno.vargas@gmail.com}
}

\maketitle

\section{Introduction}

Text Text Text Text Text Text  Text Text Text Text Text Text Text
Text Text  Text Text Text Text Text Text.
Text Text Text  Text Text \citep{Bag01}.

\section{Notes}
\subsection{Arguments}
\begin{itemize}
\item Plataforma para implementar facilmente flujos de trabajo que involucran otras librerías de 
	terceros con python API.
\item Bajada de barrera para usuarios no expertos, envolviendo flujos de trabajo.
\item Facilitar la reproducibilidad
\item In-house libraries con las tareas que más carga computacional tienen
\item Hacer un diagrama de flujo de trabajo con las forms y las funciones elementales
\item Hay unas formas nativas: MolSys con topología, trajectoria, etc.
\item Hay un Pandas dataframe codificando la topología, esto permite que la sintaxis sea más
	habitual, que en el futuro el motor de búsqueda pueda ser sustituido por alguna pandas
		optimizado o librería que trabaja con dataframes ya que se está constituyendo en un
		standard.
\end{itemize}

\subsection{Content}


\section{The computational work with biomolecular models.}

Explain the concept of lab bench.
Implementing workflows.

\section{Implementation}

Text Text Text Text Text Text Text Text
Text Text Text Text Text Text Text Text Text Text Text Text Text.


\subsection{Subsection 1}

Text Text Text Text Text Text  Text Text Text Text Text Text Text
Text Text  Text Text Text Text Text Text.

\subsubsection{Subsubsection 11}

Text Text Text Text Text Text  Text Text Text Text Text Text Text
Text Text  Text Text Text Text Text Text.

\subsection{Future implementations}

No queremos que la librería tenga un paper cada dos años, la definición de la librería es lo
suficientemente flexible para que cualquier imeplementación futura tenga cabida en la descripción
aquí hecha: nuevas formas, funciones, etc.
Text Text Text Text Text Text  Text Text Text Text Text Text Text
Text Text  Text Text Text Text Text Text.

\section{Conclusion}

Text Text Text Text Text Text  Text Text Text
Text Text Text Text Text Text  Text Text Text Text Text Text.

Hablar de hacia donde va la implementación y que puede servir de soporte para el facil desarrollo
de otras librerías.
\section*{Funding}

This work has been partially supported by the Fondos Federales project SSA XXX at Institutos de
Salud.\vspace*{-12pt}

%\bibliographystyle{natbib}
%\bibliographystyle{achemnat}
%\bibliographystyle{plainnat}
%\bibliographystyle{abbrv}
%\bibliographystyle{bioinformatics}
%
%\bibliographystyle{plain}
%
%\bibliography{Document}

\begin{thebibliography}{}

\bibitem[Bag {\it et~al}., 2001]{Bag01}
Bag,M., Name2, Name3 (2001) Article title, {\it Journal Name}, {\bf 99}, 33-54.

\end{thebibliography}
\end{document}
